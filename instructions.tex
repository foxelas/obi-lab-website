\documentclass[12pt, letterpaper]{article}
\usepackage[utf8]{inputenc}
\usepackage{hyperref}
\hypersetup{
    colorlinks=true,
    linkcolor=blue,
    filecolor=magenta,      
    urlcolor=cyan,
    }

\urlstyle{same}

\title{Instructions for update}

\begin{document}

\maketitle

\section{Preparation}
\begin{itemize}
\item Make sure the pc environment is prepared according to \url{https://foxelas.github.io/blog/jekyll/update/2021/07/24/notes-on-jekyll.html}
\item With Github, checkout to branch 'dev'.
\end{itemize}

\section{Update}
\begin{itemize}
\item To update general information, modify `\_config.yml'
\item To update the lab members, modify files: a) en$\backslash$about.html, and b) ja$\backslash$ about.html. Values for general information e.g. department name, can be recovered from `\_config.yml' as \{\{site.department[site.active\_lang]\}\}. 
\item To add new posts, add a new file in `\_posts' with filename in format `YYYY-MM-DD-yourname.markdown'. The file should start with \\

---\\
layout: post\\
title: "Website renewal!"\\
date: 2022-03-10\\
categories: general\\
lang: en\\
---\\

and then the content follows, either as plaintext, as markdown or as html. 

\item Open git bash, navigate to target folder and run \\
\$bundle exec jekyll serve
\item Inspect the site in \url{http://127.0.0.1:4000/obi-lab-website/}
\item Push changes to branch `dev'.
\end{itemize}

\section{Build}
\begin{itemize}
\item Open git bash, navigate to target folder and run \\
\$bundle exec jekyll serve
\item Inspect the site in \url{http://127.0.0.1:4000/obi-lab-website/}
\item Publish and test with github pages. Copy folder `\_site' to a different directory, checkout to branch `gh-pages' and replace its contents with the contents of `\_site' folder. Make sure the `.nojekyll' file remains intact.  Wait a bit and check the live website at \url{https://foxelas.github.com/obi-lab-website}. 
\item Fetch the contents of folder `\_site' and publish them as the new website.
\end{itemize}

\end{document}